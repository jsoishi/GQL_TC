%% Author_tex.tex
%% V1.0
%% 2012/13/12
%% developed by Techset
%%
%% This file describes the coding for rstrans.cls

\documentclass[openacc]{rstransa}%%%%where rstrans is the template name

%%%% *** Do not adjust lengths that control margins, column widths, etc. ***

%%%%%%%%%%% Defining Enunciations  %%%%%%%%%%%
\newtheorem{theorem}{\bf Theorem}[section]
\newtheorem{condition}{\bf Condition}[section]
\newtheorem{corollary}{\bf Corollary}[section]
%%%%%%%%%%%%%%%%%%%%%%%%%%%%%%%%%%%%%%%%%%%%%%%
\newcommand{\Reyn}{\mathrm{Re}}

\begin{document}

%%%% Article title to be placed here
\title{Generalized Quasilinear simulations of Taylor-Couette Flow}

\author{%%%% Author details
Jeffrey S. Oishi$^{1}$ and Morgan Baxter$^{1,2}$}

%%%%%%%%% Insert author address here
\address{$^{1}$Department of Physics and Astronomy, Bates College, Lewiston, ME USA\\
$^{2}$Company Name, Whereever, USA}

%%%% Subject entries to be placed here %%%%
\subject{xxxxx, xxxxx, xxxx}

%%%% Keyword entries to be placed here %%%%
\keywords{xxxx, xxxx, xxxx}

%%%% Insert corresponding author and its email address}
\corres{Jeffrey S. Oishi\\
\email{joishi@bates.edu}}

%%%% Abstract text to be placed here %%%%%%%%%%%%
\begin{abstract}
To do: write abstract
\end{abstract}
%%%%%%%%%%%%%%%%%%%%%%%%%%%

%%%%%%%%%% Insert the texts which can accomdate on firstpage in the tag "fmtext" %%%%%

\begin{fmtext}
\section{Introduction}
Something about Taylor (1923); GQL.

\section{Equations}

Sample equations.

\end{fmtext}

%%%%%%%%%%%%%%% End of first page %%%%%%%%%%%%%%%%%%%%%

\maketitle

\section{Runs}
Taylor-Couette flow is characterized by $\Reyn_i$, $\Reyn_o$, $\eta$, $\Gamma$. For all runs in this paper, the outer cylinder is stationary, so $\Reyn_o = 0$.

The ($\eta$, $\Gamma$) choice is motivated by \cite{1984JFM...146...45M,1984JFM...146...65M}

\begin{table}[!h]
\caption{Simulations}%%%Table caption goes here
\label{tab:simulations}
\begin{tabular}{rllllllll}%%%The number of columns has to be defined here
\hline
Run & $\Reyn_i$ & $\eta$ & $\Gamma$  & $N_r$ & $N_\theta$ & $N_z$& $\Lambda_\theta$ & $\Lambda_z$\\
\hline
A & 750 & 0.875 & 3 & 32 & 64 & 64 & - & -\\
B & 1000 & 0.875 & 3 & 32 & 128 & 64 & - & -\\
C & 1500 & 0.875 & 3 & 32 & 128 & 64 & - & -\\
\end{tabular}
\vspace*{-4pt}
\end{table}%%%End of the table


\section{Figures \& Tables}

The output for figure is:\vspace*{-7pt}

%\begin{figure}[!h]
%\centering\includegraphics[width=2.5in]{xxxxxx.eps}
%%% where xxxxxx name represents "figurename.eps"
%\caption{Insert figure caption here}
%\label{fig_sim}
%\end{figure}

\section{Conclusion}

\enlargethispage{20pt}


%\ethics{Insert ethics statement here if applicable.}

\dataccess{Insert details of how to access any supporting data here.}

\aucontribute{JSO designed the study, wrote the GQL projection operators, and implemented the original Taylor-Couette Dedalus script. MB performed the simulations, developed data analysis techniques, and added the GQL terms to the TC script. All authors drafted the manuscript.}

\competing{The authors declare that they have no competing interests.}

\funding{Who paid Morgan? Who paid me?}

\ack{We thank Ben Brown for his help validating the Taylor-Couette code. Computations for this paper were performed on the \emph{Leavitt} cluster at the Bates High Performance Computing Center.}

%%%%%%%%%% Insert bibliography here %%%%%%%%%%%%%%

\bibliographystyle{rsta}
\bibliography{TC}
 
\end{document}
